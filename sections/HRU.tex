\section{Protection in Operating Systems}
As most protection systems have different approaches to how they manage right and chains of trust they are hard to compare on equal terms.
Additionally it is often not easy to express which rights are leaked under certain conditions, or to whom they are leaked.
Oftentimes such leaks are only described informally by examining key features of a given system.

The model described in \cite{HRU} can be used to solve these problems.
By formally describing protection systems using a unified model, they can be compared in terms of the rights that they leak.
Using this approach two possible solutions (protection systems) to a problem can be compared on even terms.

In the following the model proposed by \cite{HRU} is presented.

\subsection{A Protection System}
A protection system can be described in terms of a set of rights $R$ and a set of commands $C$.
Note that both these sets are finite and static for a given system.
Because of this a system cannot dynamically add or remove rights or commands, under this model.
Dynamic creation and removal of commands can however be simulated.
\mikkel{The last part about dynamics is not part of the original article, should we expand on it?}

The state, or configuration, of a protection system is a triple $Q = (S, O, P)$, where $S$ is the set of subjects in the configuration, $O$ is the set of objects in the configuration (with $S \subseteq O$) and $P$ is an access matrix describing the rights each subject has to each object.
Thus $P[s, o]$ represents the cell in $P$ containing the set of rights that the subject $s$ has to the object $o$.
If no such rights exists we have that $P[s,o] = \emptyset$.

We can represent a configuration $Q$ by annotating the access matrix with the associated subjects and objects.
Below is an example of this approach in which there is a row for each subject and a column for each object.

\begin{figure}
\centering
\begin{tabular}{l|c|c|c|}
\multicolumn{1}{c}{} & \multicolumn{1}{c}{Sam} & \multicolumn{1}{c}{Joe} & \multicolumn{1}{c}{Data} \\\cline{2-4}
Sam & $\emptyset$ & $\emptyset$ & $\{\textbf{own}, \textbf{read}, \textbf{write}\}$ \\\cline{2-4}
Joe & $\emptyset$ & $\emptyset$ & $\{\textbf{read}\}$ \\\cline{2-4}
\end{tabular}
\caption{A representation of a protection system configuration}
\label{protection:matrixsmall}
\end{figure}

Defining a protection system is the task of defining $R$ and $C$.
Rights have no inherent semantics except for those implied by their use in commands.
A command is defined as in \cref{protection:command}.
The arguments for a command are subjects and objects; $X_i \in O \supseteq S$, however the use of these within the command come with certain restrictions.
These restrictions are a direct result of the definition of $P$ and are that $X_{s_i} \in S$ and $X_{o_i} \in O$.

\begin{algorithm}
  \DontPrintSemicolon
  \SetKwFunction{cmda}{$\alpha$}
  \SetKw{cmd}{command}
  \SetKwBlock{block}{}{end}
  \cmd \cmda{$X_1, X_2, \cdots, X_k$} \block{
    \If{$r_1 \in P[X_{s_1}, X_{o_1}] \wedge r_2 \in P[X_{s_2}, X_{o_2}] \wedge \dots r_m \in P[X_{s_m}, X_{o_m}]$}
    {$operation_1$\;$operation_2$\;\dots\;$operation_n$\;}
  }
  \caption{Command form in \cite{HRU}\label{protection:command}}
\end{algorithm}

\subsubsection{Operations}
The model described by \cite{HRU} is set to be as simple as conceivably possible.
Because of this only the list of operations in \cref{protection:operations} are allowed in commands.
No general purpose computation is directly possible using the proposed model as it does not reflect the protection aspect of the commands semantics.

The semantics of the available operations are formally specified in \cite[p. 463]{HRU}.
The operations and their associated semantics are quite self explanatory, performing simple updates of the access matrix.
Because of this they are not included here.

\begin{figure}
\centering
$\begin{array}{l l}
  \textbf{enter } r \textbf{ into } P[X_s, X_o]\quad & \textbf{delete } r \textbf{ from } P[X_s, X_o]\\
  \textbf{create subject } X_s & \textbf{create object } X_o\\
  \textbf{destroy subject } X_s & \textbf{destroy object } X_o
 \end{array}$
 \caption{The six primitive operations described by \cite{HRU}}
 \label{protection:operations}
\end{figure}

The application of an operation is written as $Q \Rightarrow_{op} Q'$ and thus $Q \Rightarrow_{op^*} Q'$ represents a sequence of operations.
An operation is executed by applying its changes to an access matrix resulting in a new access matrix.
If for instance we apply the \textbf{enter \textit{write} into $P[Joe, Data]$} operation to the access matrix described above we would have a new configuration reflecting the entered right.
The result of that operation can be seen below:

\begin{figure}
\centering
\begin{tabular}{l|c|c|c|}
\multicolumn{1}{c}{} & \multicolumn{1}{c}{Sam} & \multicolumn{1}{c}{Joe} & \multicolumn{1}{c}{Data} \\\cline{2-4}
Sam & $\emptyset$ & $\emptyset$ & $\{\textbf{own}, \textbf{read}, \textbf{write}\}$ \\\cline{2-4}
Joe & $\emptyset$ & $\emptyset$ & $\{\textbf{read}, {\color{blue}\textbf{write}}\}$ \\\cline{2-4}
\end{tabular}
\caption{The configuration from \cref{protection:matrixsmall} with an added write right}
\label{protection:matrixwithwrite}
\end{figure}

Executing a command is then a matter of applying all of its operations in sequence or doing nothing, depending on the result of the conditional expression associated with the command.
The invocation of a command (with a set of arguments) is described as $Q \vdash_{\alpha(x_1, x_2, \cdots, x_k)} Q'$ and implies the description given above.

Should we want to give \textit{Joe} write access (as above) we might do this through a command that will only allow the owner of the data to provide write access.
An example of such a command is presented in \cref{protection:conferexample}.
With such a command the \textit{write} right to an object can only be given to a subject by the owner of the object.

\begin{algorithm}
  \DontPrintSemicolon
  \SetKwFunction{cmda}{confer_{write}}
  \SetKw{cmd}{command}
  \SetKw{enter}{enter}
  \SetKw{into}{into}
  \SetKwBlock{block}{}{end}
  \cmd \cmda{owner, user, content} \block{
    \If{$own \in P[owner, content]$}
    {\enter{$write$ \into{$P[user, content]$}}\;}
  }
  \caption{Conferring write rights to another subject \cite{HRU}\label{protection:conferexample}}
\end{algorithm}
