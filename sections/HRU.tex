\section{Protection in Operating Systems}
As most protection systems have different approaches to how they manage rights and chains of trust they are hard to compare on equal terms.
Additionally it is often not easy to express which rights are leaked under certain conditions, or to whom they are leaked.
Oftentimes such leaks are only described informally by examining key features of a given system.

The model described in \citet{HRU} can be used to solve these problems.
By formally describing protection systems using a unified model, they can be compared in terms of the rights that they leak.
Using this approach two possible solutions (protection systems) to a problem can be compared on even terms.

In the following the model proposed by \citet{HRU} is presented.

\subsection{A Protection System}
A protection system can be described in terms of a set of rights $R$ and a set of commands $C$.
Note that both these sets are finite and static for a given system.
Because of this a system cannot dynamically add or remove rights or commands, under this model.
Dynamic creation and removal of commands can however be simulated.
\mikkel{The last part about dynamics is not part of the original article, should we expand on it?}
\bruno{Maybe we can add it later if we think it is important - for now i think it is fine if we just leave it like it is and put a citation on the later paper.}

\begin{definition}
The state, or configuration, of a protection system is a triple $Q = (S, O, P)$, where $S$ is the set of \subjects{} in the configuration, $O$ is the set of objects in the configuration (with $S \subseteq O$) and $P$ is an access matrix describing the rights each \subject{} has to each object.
Finally $P[s, o]$ represents the cell in $P$ containing the set of rights that the \subject{} $s$ has to the object $o$ (with $P[s, o] \subseteq R$).
If no such rights exists we have that $P[s,o] = \emptyset$.
\end{definition}

We can represent a configuration $Q$ by annotating the access matrix with the associated \subjects{} and objects.
\Cref{protection:matrixsmall} shows an example of this approach in which there is a row for each \subject{} and a column for each object.

\begin{figure}[h]
\centering
\begin{tabular}{l|c|c|c|}
\multicolumn{1}{c}{} & \multicolumn{1}{c}{Sam} & \multicolumn{1}{c}{Joe} & \multicolumn{1}{c}{Data} \\\cline{2-4}
Sam & $\emptyset$ & $\emptyset$ & $\{\textbf{own}, \textbf{read}, \textbf{write}\}$ \\\cline{2-4}
Joe & $\emptyset$ & $\emptyset$ & $\{\textbf{read}\}$ \\\cline{2-4}
\end{tabular}
\caption{A representation of a protection system configuration}
\label{protection:matrixsmall}
\end{figure}

Defining a protection system is the task of defining $R$ and $C$.
Rights have no inherent semantics except for those implied by their use in commands.

\begin{definition}
A protection system is a finite set of rights $R$ and a finite set of commands $C$.
A command is defined as in \cref{protection:command}. \stefan{jeg savner en kort uddybning af en kommando}
The arguments for a command are \subjects{} and objects; $X_i \in O \supseteq S$, however the use of these within the command come with certain restrictions.
These restrictions are a direct result of the definition of $P$ and are that $X_{s_i} \in S$ and $X_{o_i} \in O$.
\end{definition}

\begin{algorithm}
  \DontPrintSemicolon
  \SetKwFunction{cmda}{$\alpha$}
  \SetKw{cmd}{command}
  \SetKwBlock{block}{}{end}
  \cmd \cmda{$X_1, X_2, \cdots, X_k$} \block{
    \If{$r_1 \in P[X_{s_1}, X_{o_1}] \wedge r_2 \in P[X_{s_2}, X_{o_2}] \wedge \dots r_m \in P[X_{s_m}, X_{o_m}]$}
    {$operation_1$\;$operation_2$\;\dots\;$operation_n$\;}
  }
  \caption{Command form in \cite{HRU}\label{protection:command}}
\end{algorithm}

\subsubsection{Operations}
The model described by \citet{HRU} is set to be as simple as conceivably possible.
Because of this only the list of operations in \cref{protection:operations} are allowed in commands.
No general purpose computation is directly possible using the proposed model as it does not reflect the protection aspect of the commands semantics.

The semantics of the available operations are formally specified in \citet[p. 463]{HRU}.
The operations and their associated semantics are quite self explanatory, performing simple updates of the access matrix.
Because of this they are not included here.

\begin{figure}
\centering
$\begin{array}{l l}
  \textbf{enter } r \textbf{ into } P[X_s, X_o]\quad & \textbf{delete } r \textbf{ from } P[X_s, X_o]\\
  \textbf{create \subject{} } X_s & \textbf{create object } X_o\\
  \textbf{destroy \subject{} } X_s & \textbf{destroy object } X_o
 \end{array}$
 \caption{The six primitive operations described by \cite{HRU}}
 \label{protection:operations}
\end{figure}
\stefan{jeg synes ikke det er en fordel at det er i to kolonner}

The application of an operation is written as $Q \Rightarrow_{op} Q'$ and thus $Q \Rightarrow_{op^*} Q'$ represents a sequence of operations.
An operation is executed by applying its changes to an access matrix resulting in a new access matrix.
If for instance we apply the \textbf{enter \textit{write} into $P[Joe, Data]$} operation to the access matrix described in \cref{protection:matrixsmall} we would have a new configuration reflecting the entered right.
The result of that operation can be seen in \cref{protection:matrixwithwrite}.

\begin{figure}
\centering
\begin{tabular}{l|c|c|c|}
\multicolumn{1}{c}{} & \multicolumn{1}{c}{Sam} & \multicolumn{1}{c}{Joe} & \multicolumn{1}{c}{Data} \\\cline{2-4}
Sam & $\emptyset$ & $\emptyset$ & $\{\textbf{own}, \textbf{read}, \textbf{write}\}$ \\\cline{2-4}
Joe & $\emptyset$ & $\emptyset$ & $\{\textbf{read}, {\color{blue}\textbf{write}}\}$ \\\cline{2-4}
\end{tabular}
\caption{The configuration from \cref{protection:matrixsmall} with an added write right}
\label{protection:matrixwithwrite}
\end{figure}

Executing a command is then a matter of applying all of its operations in sequence or doing nothing, depending on the result of the conditional expression associated with the command.
The invocation of a command (with a set of arguments) is described as $Q \vdash_{\alpha(x_1, x_2, \cdots, x_k)} Q'$ and implies the description given above.

Should we want to give \textit{Joe} write access (as above) we might do this through a command that will only allow the owner of the data to provide write access.
An example of such a command is presented in \cref{protection:conferexample}.
With such a command the \textit{write} right to an object can only be given to a \subject{} by the owner of the object.

\begin{algorithm}
  \DontPrintSemicolon
  \SetKwFunction{cmda}{$confer_{write}$}
  \SetKw{cmd}{command}
  \SetKw{enter}{enter}
  \SetKw{into}{into}
  \SetKwBlock{block}{}{end}
  \cmd \cmda{owner, user, content} \block{
    \If{$own \in P[owner, content]$}
    {\enter{$write$ \into{$P[user, content]$}}\;}
  }
  \caption{Conferring write rights to another \subject{} \cite{HRU}\label{protection:conferexample}}
\end{algorithm}

\subsection{Analysis of Protection Systems}
Given the formal definition of a protection system we are able to ask specific questions about a given protection system.
By defining a set of requirements we can ask the same questions about several protection systems in an attempt to determine which system meets most or all of these requirements.
These questions could be any of the below:
\begin{itemize}
\item Does the system leak the right $r$?
\item Does the system leak the right $r$ to the \subject{} $s$?
\item Does the system leak the right $r$ to anyone other than \subject{} $s_1$, $s_2$ or $s_3$?
\end{itemize}
It should be noted that all these questions are answered on the basis of an initial configuration $Q$.
In other words; we are not interested in whether or not a right is leaked from an unattainable configuration.
Additionally we might be interested in determining if a right can be leaked given a specific configuration, effectively deciding what effects a specific right will introduce in the system.

It should be noted that a leak (given in the definition below) is not necessarily a negative.
Certain rights we will want the system to leak in a specific way.
We are both interested in ensuring that certain rights are not leaked and that certain other rights are.

\begin{definition}
A command $\alpha$ leaks the right $r$ from configuration $Q$ if, after running $\alpha$ on $Q$, $r$ is entered into a cell in the access matrix where it did not exist before running $\alpha$.
A protection system leaks the right $r$ given an initial configuration $Q$ if there exists a command in the system that leaks $r$ from $Q$, or there exists a configuration $Q'$ which is reachable from the initial configuration such that there exists a command in the system that leaks $r$ from $Q'$.
\end{definition}

\subsubsection{Determine if a System Leaks}
We can translate the latter two types of questions presented above into the first one.
Because of this it is sufficient to provide an algorithm that will answer this type of questions.
\Citet{HRU} provides proof that there is no generic algorithm which will answer this question.

This is the major result in the article.
It is also described how such an algorithm exists for a very simple class of protection systems called mono-operational (a system with commands with only a single operation each).
Thus an algorithm exists for very simple system and not for complex systems.

In order to answer questions about specific protection systems, such as the ones above, custom algorithms must be defined.
These algorithms must be built such that they can be applied to the protection systems being tested.
The article does however not describe how to built such an algorithm.
