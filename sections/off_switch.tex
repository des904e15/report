\section{Concerns regarding the remote access-ability of smart meters}
As the smart grid and smart meters are getting more common a great observation was made in \citet{offswitch}.
This observation is regarding the possibility of the electrical company to switch off the smart meter.
This could be if the customer is not paying.
When having smart meters controlled one place this place could be a target by terrorist, other countries or by a group of activists.
By switching off smart meters for a whole region it would be paralysed because of the necessity of electricity.


In a scenario where millions of smart meters are deployed and controlled from a central control unit without taken possible attacks into account.
\citet{offswitch} states that this is happening in the United Kingdom.
This means that there properly are a lot of vulnerabilities in the setup.
If an attacker gets access to the central control unit he has the ability to compromise all the connected smart meters.
By also changing the smart meters cryptography the attack could last for weeks for some consumers.
People would die from basic diseases as for instance hypothermia because of lack of electricity at hospitals.
The region it would effect would be in chaos.

Possible attackers for such an attack could be:
\begin{itemize}
\item Countries when there is international tension.
\item A terrorist organisation.
\item Environmental activists which are frustrated with governments not taken the environment seriously.
\item A criminal who switches off all or some smart meters from an electrical company and demands money for switching them on again.
\end{itemize}

It would also be possible to do energy theft against electrical companies.
\bruno{Skal vi have deres bud på en løsning med?}
