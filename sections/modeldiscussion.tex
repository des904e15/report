\section{Applicability in a smart meter context}
Now that the four security models have been described we will discuss how they can be applied to our project if possible.
Each of the models have different limitations and we will attempt to find the models that fit best.

Privacy is an important factor in choosing the right model.
It is necessary that it is possible to model how data is permitted to be sent between actors.


\subsection{The Bell LaPadula Security Model}
The Bell LaPadula security model works with classifications and categories because it is made for a system with multiple tiers.
Out system can be described by classifications of different granularities, so the consumer is cleared to view data of a fine granularity while the electrical can only view a coarser granularity that makes it possible for them to compute a bill.
The smart meter handles different types of data, consumption, control of appliances and connected devices.
These could each have a category that determines if the data concerns outside parties.

\subsection{Protection in Operating Systems}
Protection in Operating Systems is occupied with a different matter, namely creating a unified model that can be used to compare security models.
This is therefore more a tool for verifying that the implementation does not leak any data that it is supposed to.
It is thereby a very relevant tool in keeping a smart meter secure.

\subsection{The Chinese Wall Security Policy}
The Chinese Wall Security Policy contrasts the Bell LaPadula model by putting data in conflict classes.
Security is then achieved by placing a ``wall'' that separates individuals who access some data from all data that conflicts with that data.

Modeling our system in this way seems illogical because we cannot set up conflict classes.
Our problem is concerned with protecting data of a certain granularity from actors who do not need that particular granularity.
The Chinese Wall Security Policy will therefore be regarded as unfit for our purposes.

\subsection{Decentralized Label Model}
The Decentralized Label Model describes a way to control how data flows through a system by attaching labels to data.
This way of handling flow seems to fit our problem very well because we have data that travel from the consumer to a Data Hub and further to a multitude of actors from the Data Hub.
Being able to describe how data can flow makes this model very attractive for our purposes.
An example presenting an attempt at modeling the system was presented in \cref{dlm-example}.