The purpose of this project is to examine the security in smart meters.
Smart meters will be implemented in EU by 2022, and has to make the electrical grid more flexible.
But the speedy implementation may cause the security aspect to be less prioritized than what is desirable.

The report starts out by examining the smart meter and smart grid as well as the problems already known to exist in smart meters.
Because of the loosely defined laws in the area, as well as the ongoing implementation, the report defines a smart meter context model that will be used as reference throughout the rest of the report.

Thereafter the report will investigate available security models in order to determine if one of the already existing models can be used to model the system.
The applicability of each model will be discussed in order to aid future work on the subject.
Afterwards the smart meter context will be analyzed and possible attacks on the smart meter infrastructure will be presented in attack trees.
Attacks will be detailed and discussed and finally we will present possible paths forward.
\stefan{vi skal sikre os at vi gør dette}