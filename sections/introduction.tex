%!TEX root=../master.tex

\chapter*{Introduction}
\addcontentsline{toc}{chapter}{Introduction}

Energy consumption is ever increasing and our current electrical grid is outdated, as it is unable to handle the more dynamic production and consumption of power we have today.
There is also an increasing generation of renewable energy (e.g. wind and solar), but optimal use of this is limited by our current electrical grid.
As it is now, renewable sources are often shut down if they are producing more than can be consumed.
On the other hand, during periods with little renewable power, more expensive and environmentally unsafe sources are used.
Additionally, most suppliers only support very simple tariffs with one fixed price, or a single division of prices, throughout the day.

The implications of the above statements have led to an increased focus on energy usage, with optimizations of the electrical grid as a high priority.
This is why, by 2022 (80\% by 2020), all  electrical meters in EU member countries must be replaced by \emph{smart meters} \cite{smart_meter_survey, directive_2009_72_EC}.
An expansion of the electrical grid, as it is extended to connect more EU member countries, along with the addition of smart meters -- enabling home owners to supply themselves and others, will form a \emph{smart grid}.

The suddenness and size of this project has the potential to bring a lot of security and privacy-related issues with it.
The millions of smart meters placed in private homes throughout Europe can lead to mass privacy breaches, as smart meter measurement data mining can reveal much about each individual household.
Parties with malicious intent can use any ill-gained access to smart meters or other smart grid components to possibly shut down or in other ways control the electrical grid.

This report will have a general focus about these issues and how to secure smart meter systems.
Firstly, in \cref{context}, the context in which we operate, and possible issues, will be elaborated, concluded by some assumptions about what kind of system will be implemented -- with a base in the Danish solution.
Then in \cref{security_models} several security models will be presented, which could have a potential interest in ensuring that smart meter solutions are secure.
To get a better idea about what kind of attacks are possible in a smart meter system, \cref{attack_trees} explores possible attackers forms attack trees -- which are used to further identify concrete attack techniques.
These attack techniques are then elaborated in \cref{attack_techniques}, with a technical presentation based on current security engineering.
Finally, we conclude in \cref{conclusion} on the achievements of this report and what work from now on can be done to further the security in the coming smart meter systems.
