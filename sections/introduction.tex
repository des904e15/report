%!TEX root=../master.tex

\chapter*{Introduction}
\addcontentsline{toc}{chapter}{Introduction}

Energy consumption is ever increasing and our current electrical grid is outdated, as it is unable to handle the more dynamic production and consumption of power we have today.
There is also an increasing generation of renewable energy (e.g. wind and solar), but optimal use of this is limited by our current electrical grid.
As it is now, renewable sources are often shut down if they are producing more than can be consumed.
On the other hand, during periods with little renewable power, more expensive and environmentally unsafe sources are used.
Additionally, most suppliers only support very simple tariffs with one fixed price, or a single division of prices, throughout the day.

The statements above have led to an increased focus on energy usage, with optimizations of the electrical grid as a high priority.
This is why, by 2022 (80\% by 2020), all  electrical meters in EU member countries must be replaced by \emph{smart meters} \cite{smart_meter_survey, directive_2009_72_EC}.
An expansion of the electrical grid, as it is extended to connect more EU member countries, along with the addition of smart meters -- enabling home owners to supply themselves and others, will form a \emph{smart grid}.
