\section{The Chinese Wall Security Policy}
\mikkel{We are missing a ``definition'' of what the wall itself is. Maybe illustrate it in the figures?}
The first security models that emerged were mainly concerned with military, and were thus tailored to fit the needs of the military. 
In \citet{brewer1989chinese} a model that is relevant for commercial applications is presented.
This model fits the needs of corporate business services like an analyst that must uphold the confidentiality of his clients, meaning that he cannot advise a corporation if he has insider knowledge of a competitor.
The following description is based on \citet{brewer1989chinese}.

\subsection*{The Chinese Wall Model}
A Chinese wall is defined as the separation between what can be accessed and what cannot be accessed by a user of the system.
Information is stored in a hierarchy with three levels. 
This hierarchy is depicted on \cref{china:hierarchy}.

\begin{itemize}
	\item The lowest level contains individual items of information, stored in objects.
	\item The middle level contains company datasets that group all information that concern one company. 
	\item The highest level contains conflict of interest classes which group companies that are in competition.
\end{itemize}

\begin{figure}[h]
  \resizebox{\textwidth}{!}{
	\begin{tikzpicture}


\node[fill=black, regular polygon sides=4,inner sep=1.5pt] (v3) at (-2,0) {};
\node[fill=black, regular polygon sides=4,inner sep=1.5pt] (v2) at (-3,0) {};


\node[fill=black, , regular polygon sides=4,inner sep=1.5pt] (v4) at (-1,0) {};

\draw (v2) -- (v3) -- (v4);
\node (v7) at (-2,1) {Statoil};
\draw (v2) -- (v3) -- (v4);

\node (v14) at (0,1) {};
\node (v13) at (2,1) {Shell};
\node[fill=black, regular polygon sides=4,inner sep=1.5pt] (v10) at (2,0) {};
\node[fill=black, regular polygon sides=4,inner sep=1.5pt] (v9) at (1,0) {};



\node[fill=black, , regular polygon sides=4,inner sep=1.5pt] (v11) at (3,0) {};
\draw  (v9) -- (v10) -- (v11) ;

\node (v15) at (0,2) {Oil Companies};
\draw  (v14) edge (v15);




\node (v30) at (4,2) {};
\node (v40)at (8,2) {Banks};
\node (v23) at (8,1) {};
\node (v22) at (6,1) {Danske Bank};
\node (v24) at (10,1) {Spar Nord};
\node[fill=black, , regular polygon sides=4,inner sep=1.5pt] (v25) at (10,0) {};
\node[fill=black, , regular polygon sides=4,inner sep=1.5pt] (v28) at (9,0) {};

\node[fill=black, , regular polygon sides=4,inner sep=1.5pt] (v29) at (11,0) {};
\node[fill=black, , regular polygon sides=4,inner sep=1.5pt] (v18) at (6,0) {};
\node[fill=black, , regular polygon sides=4,inner sep=1.5pt] (v17) at (5,0) {};



\node[fill=black, , regular polygon sides=4,inner sep=1.5pt] (v19) at (7,0) {};
\draw (v17) -- (v18) -- (v19);

\draw (v23) -- (v40) -- (v30) -- (v15);
\node (v31) at (4,3) {Objects};
\draw  (v30) edge (v31);


\node at (-5,3) {\parbox{5cm}{The set of all objects}};
\draw[dashed, gray] (-3.8, 3) -- (v31);
\node at (-5,2) {\parbox{5cm}{Conflict of interest classes}};
\draw[dashed, gray] (-2.8, 2) -- (v15);
\node at (-5,0) {\parbox{5cm}{Individual data objects}};
\draw[dashed, gray] (-4.1, 1) -- (v7);
\node at (-5,1) {\parbox{5cm}{Company datasets}};
\draw[dashed, gray] (-3.4, 0) -- (v2);
\draw  (v3) edge (v7);
\draw  (v13) edge (v10);
\draw  (v22) edge (v18);
\draw  (v25) edge (v24);
\draw  (v7) edge (v14);
\draw  (v13) edge (v14);
\draw  (v22) edge (v23);
\draw  (v24) edge (v23);
\draw (v28) -- (v25) -- (11,0);
\draw (v10) -- cycle;
\end{tikzpicture}}
	\caption{The hierarchy of the Chinese Wall model.}
	\label{china:hierarchy}
\end{figure}

\paragraph{Example} A conflict of interest class could be \emph{Oil companies} which contains \emph{Statoil} and \emph{Shell}. \emph{Statoil} contains objects of information that could compromise the company if \emph{Shell} would obtain the information.
\mikkel{I suggest that we present this example as a figure instead of \cref{china:hierarchy}.}
\bruno{+1}

\subsection{The Security Policy}
\mikkel{I think we should use the concrete examples from above instead of ABC for the various classes/companies}
The concept of the Chinese wall security policy is that a \principal{} that has access to information about company A which resides in a conflict of interest class C cannot access information about company B if B is also in conflict of interest class C.

In \cref{china:conflict} two companies are in the same conflict of interest class and a \subject{} will not be able to access both of them at the same time.
Initially the \subject{} will be able to choose freely from the two possibilities, but after choosing one of the companies the other will be unaccessible due to the conflict of interest.

\begin{figure}[h]
  \centering
\resizebox{0.3\textwidth}{!}{  
  \begin{tikzpicture}


\node (v3) at (-2,0) {};
\node (v2) at (-3,0) {};
\node (v1) at (-3,-1) {};
\node (v6) at (-2,-1) {};
\node (v4) at (-1,0) {};
\node (v5) at (-1,-1) {};
\draw (v1) -- (v2) -- (v3) -- (v4) -- (v5);
\node (v7) at (-2,1) {A};
\draw (v6) -- (v3) -- (v7);



\node (v14) at (0,1) {};
\node (v13) at (2,1) {B};
\node (v10) at (2,0) {};
\node (v9) at (1,0) {};
\node (v8) at (1,-1) {};

\node (v12) at (3,-1) {};
\node (v11) at (3,0) {};
\draw (v8) -- (v9) -- (v10) -- (v11) -- (v12);
\draw (2,-1) -- (v10) -- (v13) -- (v14) -- (v7);
\node (v15) at (0,2) {};
\draw  (v14) edge (v15);
\end{tikzpicture}
  }
  \caption{Two companies in the same conflict of interest class}
  \label{china:conflict}
\end{figure}
definition
In \cref{china:conflict2} two companies are added to the system in a new conflict of interest class.
A \subject{} will be able to access one company from each class.
After choosing a company in class C, the \subject{} still has a free choice from the two companies in class D.

\begin{figure}[h]
  \centering
  \resizebox{0.8\textwidth}{!}{
    \begin{tikzpicture}


\node (v3) at (-2,0) {};
\node (v2) at (-3,0) {};
\node (v1) at (-3,-1) {};
\node (v6) at (-2,-1) {};
\node (v4) at (-1,0) {};
\node (v5) at (-1,-1) {};
\draw (v1) -- (v2) -- (v3) -- (v4) -- (v5);
\node (v7) at (-2,1) {A};
\draw (v6) -- (v3) -- (v7);



\node (v14) at (0,1) {};
\node (v13) at (2,1) {B};
\node (v10) at (2,0) {};
\node (v9) at (1,0) {};
\node (v8) at (1,-1) {};

\node (v12) at (3,-1) {};
\node (v11) at (3,0) {};
\draw (v8) -- (v9) -- (v10) -- (v11) -- (v12);
\draw (2,-1) -- (v10) -- (v13) -- (v14) -- (v7);
\node (v15) at (0,2) {C};
\draw  (v14) edge (v15);




\node (v30) at (4,2) {};
\node (v40)at (8,2) {D};
\node (v23) at (8,1) {};
\node (v22) at (6,1) {K};
\node (v24) at (10,1) {J};
\node (v25) at (10,0) {};
\node (v28) at (9,0) {};
\node (v27) at (9,-1) {};
\node (v26) at (10,-1) {};
\node at (11,-1) {};
\node (v29) at (11,0) {};
\node (v18) at (6,0) {};
\node (v17) at (5,0) {};
\node (v16) at (5,-1) {};
\node (v21) at (6,-1) {};
\node (v20) at (7,-1) {};
\node (v19) at (7,0) {};
\draw (v16) -- (v17) -- (v18) -- (v19) -- (v20);
\draw (v21) -- (v18) -- (v22) -- (v23) -- (v24) -- (v25) -- (v26);
\draw (v27) -- (v28) -- (v25) -- (v29) -- (11,-1);
\draw (v23) -- (v40) -- (v30) -- (v15);
\node (v31) at (4,3) {};
\draw  (v30) edge (v31);
\end{tikzpicture}
    }
  \caption{Introducing a conflict of interest class}
  \label{china:conflict2}
\end{figure}
This policy is being upheld by two properties.
These correspond to the two identical properties defined in Bell Lapadula \stefan{ref til definitionen i Bellapadula}.

\begin{definition}\label{def:contradiction}
A \emph{contradiction} is a proposition that is always false for any value of its variables.
\end{definition}
\paragraph{Simple Security}

When a \principal{} S requests to read an object O, this can only be permitted if one of the following requirements are fulfilled:

\begin{itemize}
\item O is in the same company dataset as an object already accessed by S (the object is within the wall).
\item O belongs to a different interest class than any of those S has previously accessed.
\end{itemize}

\paragraph{*-Property}

When a \principal{} S requests to write to an object O if the following requirements are both fulfilled.

\begin{itemize}
\item Access to O is permitted by the simple security rule.
\item No objects from another object P in another company dataset D, which contains unsanitized information can be read when requesting write access to O.
\end{itemize}

The second requirements ensures that unsanitized information cannot leave the company dataset, but it is still possible to sanitize the data in order to compare companies.
