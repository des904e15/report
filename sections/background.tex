%!TEX root=../master.tex
In order for us to discuss potential problems with any smart meter system, we first need some background knowledge.
In this chapter we will first of all very shortly present what the smart grid and smart meters are, along with a short outline of potential problems.
Two big issues, privacy and remote access-ability, will be further elaborated as these are two of the biggest issues with any smart meter system.
Finally, some assumptions will have to be made about any future implementations in order to more concretely discuss any security-related problems and/or solutions, which is done by presenting a ``Smart meter context model''.

\section{Smart Grid and Smart Meters}\label{smart_grid_smart_meter}
A smart grid is an electrical grid supported by a net, allowing two-way communication, whereas earlier it was only one-way.
This allows for much more dynamic power supply and consumption, as suppliers will know more about their consumers, and consumers will have more options in regards to their consumption.

The outline of a smart grid can be seen in \cref{fig:background:smartgrid}.
This consists of three main actors:

\begin{itemize}
	\item Power suppliers such as Windmills, solar panels, traditional power plants and external suppliers.
	\item A net supplier such as a Datahub \footnote{Danish term from \cite{LOV_nr_575_af_18-06-2012}} and smart meters.
	\item End consumers such as smart homes/buildings.
\end{itemize}

\begin{figure}[H]
	\includegraphics[width=\textwidth]{figures/SmartGrid_Ueberblick_ohneLegende.jpg}
	\caption{Smart grid outline\protect\footnotemark}
	\label{fig:background:smartgrid}
\end{figure}
\footnotetext{\url{https://www.telekom.com/medien/bild-ton-und-infografiken/infografiken/155030}}

A smart grid has several advantages over the old grids \cite{smartgrid_gov, directive_2009_72_EC}.
For instance it will enable prices to be base on supply and demand as well as the power source available.
Consumers will also be able to connect their own power generators such as windmills and solar panels.
Prices will then more realistically reflect environmental strain which can make the consumer more likely to consider the environment when consuming power.

The use of smart meters will also provide more detailed information about the power consumption of the consumer which enables the consumer to use smart appliances that can turn on when the power is cheap.
The consumer can also monitor the power usage and adjust his practices according to the prices.

On a large scale the smart grid also enables connecting the national grids across Europe for better utilization of the generated power.


\subsection{Smart appliances}
\label{background:smart_appliances}
Smart appliances\cite{smart_appliances} are ordinary appliances, such as air conditioners or dishwashers.
What makes them smart is the fact that the times or periods in which they are turned on are highly configurable, so that they can match changing power prices.
This can be done manually, by the user looking at current pricing and setting a timer when to turn on.
However, some smart appliances can also by themselves look up prices and choose appropriate times to turn on.

\subsection{Problems}
In regards to enabling an EU-wide smart grid, there are bound to be problems, as this is an immense project.
The roll-out will not be the same in every nation, as each nation has its own infrastructure and legal constraints.
Additionally they differ in which parts of their electrical grids are privately and publicly owned.
However, some shared problems still exist, such as privacy, conflicts of interest, and attack vulnerabilities \cite{offswitch, smart_meter_survey, security_economics}.

\subsubsection{Different architectures}
Even though the implementation of smart grid and smart meters is a EU project, they are not that specific when it comes to where responsibilities should lie.
It is up to each nation, depending on their current system and infrastructure, who should distribute smart meters and who will own the data.

Here are a few examples of different architectures \cite{smart_meter_survey}:
\begin{itemize}
	\item \textbf{Italy} - Regulated monopoly (supplier and distributor is the same).
	\item \textbf{Germany} - Free market for both distributor and supplier, households will have free choice in both.
	\item \textbf{UK} - Centralized government-licensed monopoly, which will have a datahub from which data will be distributed to suppliers, customers, etc.
	\item \textbf{Denmark} - Centralized through 60 government-regulated distributors, a government-owned datahub contains all data \cite{dk_marked,LOV_nr_575_af_18-06-2012}.
\end{itemize}

\subsubsection{Privacy}
With the adaptation of smart meters it will be possible to collect power usage data more often.
This is possible as it can be done remotely, whereas earlier a display would have to be read on the physical electrical meter.
It also makes sense to have more readings, as tariffs will vary more \cite{directive_2009_72_EC, eu_smart_meter_pricing}, and therefore more measurements are needed to match prices during tariff-determined periods.

This possibility of finer granularity meter readings can be exploited in several ways and by different actors.
The supplier would like this information in order to better tailor prices for a certain customer, thus limiting competition as their competitors do not possess this information.
If the power usage data was obtained by an electronics vendor, they could use it to target certain advertisements, based on what devices they detect (or do not detect) through the power usage.
Anyone with malicious intent could use the power data to determine who, if anyone, is home (e.g. burglars).
Finally, the government could use this data to surveil the public.

Some important issues with smart meters and their measurements are therefore:
\begin{itemize}
	\item Who owns the data?
	\item Who should be able to access the data?
	\item With which granularity should supplier/government/user be able to access the data?
	\item How do we ensure that only the correct people have access to the smart meter and its data?
\end{itemize}

\subsubsection{Conflict of interest}
The various actors involved with smart grids have different interests and potential gains.
Governments generally want lower consumption for environmental reasons.
They especially want consumption down during peak hours, in order to avoid the usage of environmentally unfriendly energy sources.
Depending on tariffs, suppliers might also want power consumption down during peak hours if they cannot provide enough, from their perspective, cheap energy.
However, suppliers generally want high consumption, so that they can sell more power and thus make more money.
Consumers want to use power as needed, but at lower costs.

These interests exemplify considerations by actors in a smart grid system.
Naturally there could be more such as consumers and suppliers having environmental considerations or governments seeking to increase productivity on a national level with lower regards for the environment.

The above relations spark several conflicts of interest:
\begin{itemize}
	\item Should governments be able to control user power consumption? If so, how?\footnote{E.g. carbon rationing in the UK \cite{security_economics}}
	\item Should others than the user be able to turn off the power? (See \cref{off_switch}) If so,
	\begin{itemize}
		\item who should be able to use this?
		\item how should it work? (With abuse in mind.)
	\end{itemize}	
\end{itemize}

\subsubsection{Vulnerabilities}
In the current system consumeres are already tampering  with their mechanical electrical meters.
Turning these mechanical meters into smart meters will possibly remove attacks, but will definitely open up for new attacks.
Switching to smart meters introduces a new branch of attacks; digital attacks.
These include attacks that are general to any publicly exposed unit, or any unit that sends data over public networks.
