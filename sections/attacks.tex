%!TEX root = ../master.tex
\section{Network attacks}\label{attacks}

\bruno{intro til network attacks \#63}

\subsection{Replay attacks}\label{replay_attack}
This description of replay attacks is based on \citet[p.~223]{cryptoenginering}.

Eve can perform a replay attack if she is able to record a message from the communication between Alice and Bob.
She can then send the message to Bob at a later point in time. 

A related concept is retries.
During the run of a protocol Alice does not receive a response from Bob.
This could be because Bob did not receive Alice's request or a similar network problem.
To solve this Alice sends her message again.
This message is known as a retry.

Now Bob can receive both a replay from Eve and a retry from Alice.
Bob needs to deal with this properly so Eve cannot abuse replays to her advantage.
Abusing replays is known as a replay attack.

\subsection{Packet injection}
This description is inspired by \cite{packetinjection}.
If Eve can intercept communication between Alice and Bob she can inject packages into the stream of messages without either party knowing.
A packet is injected by constructing a packet with the source address as one of the endpoints, Alice or Bob.
In this way it looks like Alice or Bob originated the packet, and it seems like part of the communication.
By abusing this it can be possible to affect the communication of Alice and Bob.
Eve could be sending packets that simulate a protocol error, in order to disrupt the communication between Alice and Bob.
Or, by injecting new packets, Eve can alter the course of the communication, which can be dangerous if the communication involves sensitive information.

\section{Side channel attacks}
This description of side channel attacks is based on \citet[p.~132]{cryptoenginering}.

Side channel attacks is the class of attacks that take advantage of a alternate channels of information.
This could be how much time an operation takes, the power consumption during the operation or magnetic fields.

There are means for protecting against this type of attacks, but it is hard to eliminate information leakage from all possible channels.

\subsubsection{Differential Power Analysis}
One particular side channel attack is differential power analysis described in \citet{DPA}.
The article attacks a device encrypted with DES by recording power traces of the device performing encryption operations.
Because of the way semiconductors work, different operations will have a discernible signature on the power trace.
By recording enough traces information about the key can be derived from properties of the trace, and with enough traces the key can be derived.
\stefan{nok om DPA?}

\stefan{Hvis vi vil have mere om side channels: http://www.rambus.com/timing-attacks-on-implementations-of-diffie-hellman-rsa-dss-and-other-systems/}

\section{Buffer Overflow}
The following is based on \citet[p. 18]{foster2005buffer} and \citet[Section 1.1]{ruwase2004practical}.

A buffer overflow is an attack on a vulnerable program which modifies its memory state.
This gives the attacker the opportunity to control the machine where the program is executed with the same rights as the program(for instance ``root'').

A vulnerable program is a program that does not check if a given input exceeds its buffer size in memory.
If the program gets some input which exceeds its memory buffer size the program will copy the rest of the input in some adjacent buffer.


\section{Social engineering}
\bruno{intro \#64}
\subsection{Phishing}

The following is based on \citet{security_engineering_ross_anderson} and \citet{dhamija2006phishing}.
Phishing is the process of tricking a user to use a malicious website and by that to extract some sensitive information.
This typically happens by inserting a link in an email and send it to the user.
The tricky part about phishing is to convince the user that the email, with the link and the link itself, is legit.

An example could be a fake email to get a users credentials for PayPal.
The email would contain the logo from PayPal and would tell the user to update his account information.
If the user then uses the link from the email he will get transferred to a malicious site.
When the user then enters his user name and password and submits it to the site the attackers will have his credentials.
To make it even worse for the user the attacker will redirect the user to real PayPal website and the user will never know what happened.

\section{Malware}
\bruno{\#34}
\subsection{Backdoor}
\bruno{\#37}
