%!TEX root = ../master.tex
\subsection{Attacks} \stefan{skal nok have et mere sigende navn afhængig af hvad det kommer til at indeholder}

\stefan{intro til attakcs}

\subsubsection{Replay attacks}\label{replay_attack}
This description of replay attacks is based on \citet[p.~223]{cryptoenginering}.

Eve can perform a replay attack if she is able to record a message from the communication between Alice and Bob.
She can then send the message to Bob at a later point in time. 

A related concept is retries.
During the run of a protocol Alice does not receive a response from Bob.
This could be because Bob did not receive Alice's request or a similar network problem.
To solve this Alice sends her message again.
This message is known as a retry.

Now Bob can receive both a replay from Eve and a retry from Alice.
Bob needs to deal with this properly so Eve cannot abuse replays to her advantage.
Abusing replays is known as a replay attack.


\subsection{Side channel attacks}
This description of side channel attacks is based on \citet[p.~132]{cryptoenginering}.

Side channel attacks is the class of attacks that take advantage of a alternate channels of information.
This could be how much time an operation takes, the power consumption during the operation or magnetic fields.

There are means for protecting against this type of attacks, but it is hard to eliminate information leakage from all possible channels.

\subsubsection{Differential Power Analysis}
One particular side channel attack is differential power analysis described in \citet{DPA}.
The article attacks a device encrypted with DES by recording power traces of the device performing encryption operations.
Because of the way semiconductors work, different operations will have a discernible signature on the power trace.
By recording enough traces information about the key can be derived from properties of the trace, and with enough traces the key can be derived.
\stefan{nok om DPA?}

\stefan{Hvis vi vil have mere om side channels: http://www.rambus.com/timing-attacks-on-implementations-of-diffie-hellman-rsa-dss-and-other-systems/}