%!TEX root=../master.tex

As was seen in the previous chapter, the implementation of smart meters can lead to problems related to both security and privacy.
There are many stakeholders who should, and others who should not, in some way or another be able to access the individual consumer's smart meter.
Therefore we explore the possibility of using established security models to ensure security or privacy is not violated.
The security models will be presented in chronological order of publication, in order to show the development of the models through time.

Firstly, we will present the Bell-LaPadula Security Policy, which is a military-developed model mainly concerned with controlling information flow and authorization.
Then we will present a more general approach of describing protection systems, so that it based on this description can be showed whether the described system is secure or not.
As an alternative to Bell-LaPadula we will present The Chinese Wall Security Policy, which is a commercial security model concerned with more dynamic authorization.
Finally, we present a more modern security model with the Decentralized Label Model, with main concerns similar to Bell-LaPadula.
