\section{Smart meter privacy concerns}\label{smart_meter_privacy}
The consumption data collected by a smart meter can be fine-grained.
This fine-grained data means that sensitive knowledge about the consumer can be obtained.
The following describes how one can obtain and use this knowledge, which is based on \citet{privacy_memoir}.
\mikael{I think there should be something about how this knowledge about the consumer could be exploited.}

\subsection{Experiment}
The setting, in which is collected sixty days of power consumption data from three households, consists of the following physical components:
\begin{itemize}
\item TED energy monitor
\item SheevaPlug
\end{itemize}
The energy monitor is connected to the two incoming electricity phases, A and B.\cite{TED_installation_guide}
The SheevaPlug is connected to the router, and the energy monitor, and is used to access the data remotely.
The energy monitor returns a tuple $(t,p)$ where $t$ is the time and $p$ is the power-usage since last measurement.
The power is measured every second, so we have a tuple for every second with the power used the past second.
Finally, the three households are to make a "power activity"-journal for a minimum of three days.
In this journal they write down when they turn on and off any electric devices.

When the data is collected it is analysed in four steps:
\begin{itemize}
\item Pre-process data
\item Tag power events
\item Filter out automated appliances
\item Map consumption events to real life events
\end{itemize}

\paragraph{Pre-process data}
The data is pre-processed using DBSCAN, which is a density based clustering algorithm.
This helps group power tuples into power segments, where a power segment is a collection of power tuples with a pattern adjacent in time.
Each power segment then gets tagged with a label, start time, average power usage, duration, beginning power step, and a shape label.

\paragraph{Tag power events}
In \cref{consumption_one_day} we are able to see the consumption data for one household after the second step of the analysis.
On the x-axis we have time in hours, for an entire 24-hour period, and on the y-axis we have power usage in kWh.
Already at this stage of the analysis we are able to say something about when there is activity in the household.
We can also see how there are some automated appliances that are on all the time or at certain intervals.
We can determine this by assuming there is almost no activity in the night, and by collecting data over several days.


\begin{figure}
  \begin{center}
    \includegraphics[width=\textheight, angle=90]{consumption_one_day.png}
  \end{center}
  \caption{A day of consumption data after the second step of analysis, labelled with real life events -- taken from the power activity journal.}
  \label{consumption_one_day}
\end{figure}

\paragraph{Filter out automated appliances}
The third step is to filter out automated appliances, such as the refrigerator (see \cref{consumption_one_day}).
This is done by looking at the consumption data and reason about whether the power usage is from a human interaction or not.
If not for instance at night this is probably an automated appliance and by looking a lot of data one can be rather certain.
At last the power segments are label to real time events by looking at the power activity journals.

\paragraph{Map consumption events to real life events}
In \cref{detailed_consumption} we can see how the consumption for a small time period looks like after four steps of analysis.
At this point it is pretty easy to say something about what kind of activities are going on.
The time period is for a typical morning where we can see the household uses the stove, the coffee maker, the toaster and some computer screens.

\begin{figure}
  \begin{center}
    \includegraphics[width=\textwidth, angle=90]{detailed.png}
  \end{center}
  \caption{Power consumption data after four steps of anĺysis for a small time period some morning when the household is getting breakfeast etc..}
  \label{detailed_consumption}
\end{figure}


As we can see by this example of collecting a small amount of consumption data and doing trivial analysis it is already possible to say alot about a household.
Utilities or alike that have access to much more data can learn even more by combining their knowledge.

\subsection{Privacy concerns}\label{privacy_concerns}
This leads to privacy concerns, where some of them are listed below:
\begin{itemize}
  \item Amount of people home
  \item Household activities
  \item Information about apppliances
\end{itemize}

When having a lot of consumption data one can reason about the amount of people home and can begin to see patterns in the power usage and map that to people.
This means over time it is possible who is home, where who is a power usage pattern for a person in the household.

It is also possible to say a lot about houseohld activities even with a small amount of consumption data.
For instance we can begin to reason about if one got a good nights sleep or if one was eating hot or cold breakfast.
And even if the household was watching a specific sport event last night.

By extending the analysis one could add to look for specific appliance signatures in the power trace and by that say some more about the household.\cite{NILM}
