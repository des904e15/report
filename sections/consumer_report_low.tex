%!TEX root=../master.tex

\section{Reporting a low power consumption}
To facilitate the reader in understanding the details of the attack tree in \cref{report_power_attack_tree} its nodes and sub-\emph{``trees''} will be described below.
In order for an electrical company to bill the consumer, he must first receive information from the smart meter about the consumers power consumption.
The root node splits the tree into two different possible attacks that could achieve this goal.
These attacks will be described below, for more technical descriptions of the specific attacks see \cref{attacks}.

\subsection{File false complaint against distributor}
This attack involves the attacker attempting to prove that the electrical company is billing him for a too large power consumption.
This despite the bill actually corresponding with correct usage and price information.

The project group suggest two different approaches to achieving this goal:
\begin{enumerate}
  \item The user develops an application that mimics an official app describing the power consumption.
  This application will display an incorrectly low information about the consumers power consumption.
  \item The user successfully alters pricing information as received from the electrical company.
  He will want to lower the price locally, such that when he examines his ``expected bill'' the total price will be lower than that billed by the electrical company.
\end{enumerate}
Both of the above require the user to claim wrong-doing on behalf of the electrical company.
Any one individual might have a problem in assuring the validity of their claim.
However applying this attack on a larger scale will provide a possibly stronger case.

Additionally this type of attack might be carried out by the attacker on other consumers' smart meters, effectively hiding his intent by having multiple unaware consumers complain as well.

\subsection{Falsify sent comsumption}
As the attacker (the consumer) is interested in what is reported to his electrical company, he could falsify this information.

He would intercept the packages sent from the meter, modify them and then resend them.
This approach will either require the data to be unencrypted, which does seem unlikely, or that the attacker has enough information about the system to decrypt and re-encrypt the packages.

Alternatively the attacker might intercept packages at times where he knows his power consumption to be low and resend them at a later point in time where he knows them to be high.
This is known as a replay attack (see \cref{replay_attack}).
