%!TEX root = ../master.tex
\section{Smart meter context model}

\begin{figure}[h]
  \includegraphics[width=\textwidth]{system}
  \caption{The smart meter system\cite{tdlm}}
  \label{contextual:system}
\end{figure}

The system is pictured in \cref{contextual:system}.

\paragraph{Data connections}
The lines indicate information flow between actors in the system.
The consumer (\textit{Household Owner/Residents}) can interact with the smart meter in order to control smart appliances and monitor the production of his home production devices.

The \textit{Data Hub} stores the comsumption data of the consumer.
This data is available for the consumer, the \textit{Distribution Companies} and the \textit{Electrical Company}.
These three actors have permission to see the consumption data in different granularities.

The consumer can see the data in the finest granularity, whereas the distribution companies and the electrical companies has accesss to a coarser granularity that makes it possible for them to calculate the bill.

\textbf{Note:} The distribution companies send price information to the data hub (not pictured), which the consumer then downloads to his smart meter.

\subsection{Actors and threats}
Placing a smart meter in a consumers home and simultaneously requiring external access to its data is a complicated task with many potential risks.
\Cref{contextual:sm_model} represents this context and the actors therein.
This will be used as an offset for a brainstorm that maps the potential attacks on the system.

The actors represented in \cref{contextual:sm_model} are not representative of the complete set of actors having some interaction with the system.
However these other actors communicate with the system through the smart meter and to not clutter the visual representation they have not been included.
They will still be listed as actors below and possible threats to/from them will be considered.

\begin{figure}[h]
  \centering
  \includegraphics[width=1\textwidth]{smart_meter_context_model}
  \caption{The smart meter context model}
  \label{contextual:sm_model}
\end{figure}

\paragraph{Actors}
Below is the listing of the actors represented in the full system and context.
As noted above, some of these actors are not represented visually in \cref{contextual:sm_model} but are part of the information flow of the entire system (see \cref{contextual:system}).
The listings below will not describe the potential threats in the system, but only list the possible actors involved in such threats.
\begin{itemize}
\item \textbf{Consumer}\\
The resident of the depicted home and the one which power consumption is monitored.
The smart meter is installed in the consumers home.

\item \textbf{Neighbour}\\
The next door neighbour to the consumer.
This person can also be viewed as a consumer and be expected to have a situation similar to the consumers.
However the neighbour is viewed only in the context of living near the consumer.
Typically the neighbour will perform attacks on the consumer in order to remove annoyances, such as loud appliances or music.
In case of neighbourly disputes, messing with the SM and connected appliances could also be viable strikes.
\item \textbf{Partner}\\
The consumer's partner (or significant other).
It is assumed that this is not a person that is living with the consumer.
The partner would like to surveil or get revenge over the consumer in case of misdoings on the consumer's part, such as adultery.
\item \textbf{Burglar}\\ wants to find out when the consumer is home in order to break in, what products the consumer own in order to assess him as a target

\item \textbf{Electrical Company}\\
The company billing the consumer for his power consumption.
\item \textbf{Power production company}\\
Produces the power and supplies it to the power grid.
This power production can be in the form of power plants or renewable energy.
\mikael{Is this guy important?}
\item \textbf{Distribution Company}\\
Manages the part of the grid closest to the consumer and provides him with electricity.
Installs the smart meters in the costumers home.
\item \textbf{Government}\\
The government officials (including counties and municipalities).

\item \textbf{Developers of third party apps}\\
the developer of apps that the consumer will use for seeing his electrical consumption. Can be both mobile, desktop and web applications
\item \textbf{Smart meter manufacturer}\\
\item \textbf{Appliance company}\\ the provider of home appliances that can collaborate with the SM.
\end{itemize}

\paragraph{Objects}
\begin{itemize}
\item \textbf{Smart meter}\\ controls home appliances and also records consumption for each.
\item \textbf{Appliance}\\ has an option to be controlled from the smart meter. For instance a washing machine can be started and stopped when scheduled.
\end{itemize}

\subparagraph{Smart Meter}
The smart meter is installed in the household of the consumer, in this case a house.
Home appliances are connected to the SM through their power cable, which provides electricty and information exchange.
The SM can be accessed through an API with different levels of rights, depending on the actor.
The consumer can access and manage his home appliances connected to the smart meter through some sort of program for instance an app, developed by a third party.
The partner has access to the SM by having a password from the consumer or having her own account.
The distribution has the ability to update software, check the consumption and switch off the SM.
The home appliance company can update the firmware of their appliances through the SM.
The burglar and the Neighbour can try to access the SM through the API or physically at the household.
