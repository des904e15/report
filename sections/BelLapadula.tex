\section{Bell Lapadula}

Bell and LaPadula have devised a mathematical model that targets military and governmental computer systems.
The model formalizes users accessing data and how to handle this in a secure way so a confidential file cannot be leaked to a lower classification level.
The following is based on \citet{lapadula1996secure}.

\subsection{Access attributes}
The model considers four attributes for access in a complex computer system: \emph{read-only}, \emph{append}, \emph{execute} and \emph{read/write}.
In addition \emph{control access} is used for giving out attributes to other users.

\paragraph{Read-only}
This attribute makes it possible to read the object but not alter it.
The classical example is a file that contains information that should not be changed.
An example of this could be a list containing the \principals{} in the system with their clearance levels.
A user of low clearance should be able to read this list but not change it.

Another case is input devices.
A card reader does not have content in iteself, so it could be modeled as a read-only object.

\paragraph{Append}
Append describes a pure write operation.
This means that it is possible to append information to the end of a file without being able to extract information about the rest of the file.

This can also be used with a printer which appends information what is being printed.
By doing this it is sufficient that the classification of a piece of information is matching the classification of the printer in order to prevent unauthorized personnel from reading the information.

\paragraph{Execute}
The exection attribute makes it possible to execute an executable file.
If the \principal{} does not have permission to read or write the file he will only be able to execute it.
Similarly the executable file can produce output that is of a higher classification level than the clearance level of the \principal{} executing it.

\paragraph{Read/write}
This attribute is an interactive read and write access and is what is traditionally used when editing files.

\paragraph{Control access}
The control access attribute models the notion of a \principal{} having control over a file.
Having this attribute a principal{} can give out the four attributes just described to other \principals{} in the system.

\subsection{Requests and decisions}

\subsection{The mathematical model}

Security condition

*-property

Proofs and rules?

