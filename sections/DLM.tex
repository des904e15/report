\section{Decentralized Label Model}
\bruno{Skal vi holde os til et term: value, data, information. De bruger det i flæng i litteraturen?}
\newcommand{\xvalue}{value}
\newcommand{\xvalues}{values}

\paragraph{Acts for relation}
The acts for relation, $\rightarrow$, can be used to model groups and roles.
To model two \principals{}, Amy and Bob, to act for a group of programmers: $Amy \rightarrow programmers$ and $Bob \rightarrow programmers$.
Or if Amy has different roles: $Amy \rightarrow Amy\_programmer$ and $Amy \rightarrow Amy\_admin$.

\paragraph{Value}
A \xvalue{} is a piece of data in a program, for instance an integer.

\paragraph{Privacy policy}
A privacy policy is represented as an owner of some \xvalue{} and a set of readers.
The syntax is: <owner>: <readers>.
The owner is the \principal{} who owns the \xvalue{} that was used for constructing the \xvalue{}.
The readers are the \principals{} allowed by the owner.
The owner is implicitly allowed to read his own data.

\paragraph{Labels}
A \xvalue{} is associated with a label and is a set of privacy policies.
The policies are intersected, so we have one set of owners and readers.
When the \xvalue{} flows through the system, all the policies need to be obeyed.
Labeled \xvalues{} are only released by the consensus of all the owners and can only be read by the readers.
If another privacy policy is added it restricts the \xvalue{}.

\paragraph{Relabeling}


\subsection{Limitations}
Timing attacks on timing channels such as cache miss timing.
