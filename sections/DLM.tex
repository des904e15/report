\section{Decentralized Label Model}
\bruno{Skal vi holde os til en term: value, data, information. De bruger det i flæng i litteraturen?}
\newcommand{\xvalue}{value}
\newcommand{\xvalues}{values}

\paragraph{Acts for relation}
The acts for relation, $\rightarrow$, can be used to model groups and roles.
To model two \principals{}, Amy and Bob, to act for a group of programmers: $Amy \rightarrow programmers$ and $Bob \rightarrow programmers$.
Or if Amy has different roles: $Amy \rightarrow Amy\_programmer$ and $Amy \rightarrow Amy\_admin$.

\paragraph{Value}
A \xvalue{} is a piece of data in a program, for instance an integer.

\paragraph{Privacy policy}
A privacy policy is represented as an owner of some \xvalue{} and a set of readers.
The syntax is: <owner>: <readers>.
The owner is the \principal{} who owns the \xvalue{} that was used for constructing the \xvalue{}.
The readers are the \principals{} allowed by the owner.
The owner is implicitly allowed to read his own data.

\paragraph{Labels}
A \xvalue{} is associated with a label and is a set of privacy policies.
When the \xvalue{} flows through the system, all the policies need to be obeyed.
The policies are intersected, so we have one set of owners and readers.
Labeled \xvalues{} are only released by the consensus of all the owners and can only be read by the readers.
If one or several privacy policies are added it restricts the \xvalue{}.
A label with an empty set of readers means that all readers are allowed.
If a \principal is not among the owners of a label, it is the same as if it was added as a privacy policy with all posible readers.
When we have three owners: $o_1, o_2, o_3$ and three readers: $r_1, r_2, r_3$ and the following labels:
$L_1 = {o_1: r_1,r_2, o_2: r_1, r_3}$ $L_2 = {o_1: r_1,r_2, o_2: r_1, r_3, o_3: r_1, r_2, r_3}$ we have that: $L_1 = L_2$ because when $o_3$ is not among the owners in $L_1$ it is the same as if it was there with all possible readers as in $L_2$.


If a label contains no policies it is written as an empty set.
A label can contain one or more privacy policies for th same \principal{}.
These are enforced just as other policies.
\bruno{You want more example??}


Some further notation:
We have policy $K$ and label $L$.
\begin{itemize}
\item $K \in L$, policy $K$ is a part of label $L$
\item $o(K)$ or $oK$, the owner of policy $K$
\item $r(K)$ or $rK$, the set of readers of policy $K$
\end{itemize}
\bruno{Kommer lige an på hvad vi ender med at bruge}

\subsection{Rules}
This section contains the rules that need to be followed to avoid information leakage.

\paragraph{Relabeling by Restriction}
This rule is for when one is doing an assignment of a \xvalue{} into a variable.
When a \xvalue{} is read its label is relabelled to the variable and forgotten.
To avoid leakage this relabeling process the label of the varable must be at least as restrictive.

When we have $L_1 \sqsubseteq L_2$ it means that $L_2$ is more restricted than $L_1$.

\subsection{Limitations}
Timing attacks on timing channels such as cache miss timing.
