%!TEX root=../master.tex

\newcommand{\policy}[2]{\ensuremath{#1\!:\;#2}}

\section{Decentralized Label Model}
The Decentralized Label Model (DLM) is a modern information flow control model devised in 1999 \cite{myers1999mostly}.
As the name suggests, it revolves around labels (similar to previous security models), however, DLM is decentralized.
This means that it can be applied to systems with no trusted third party, or even any trust throughout the system.
By attaching security labels to objects in code, it can be controlled how information should be shared throughout the code and at code-endpoints (input/output to/from other programs).
It is possible to do both static and run-time checking of labels.
The final major point of DLM is that it is formalized and even though no implementation is supplied, it should be applicable to existing programming languages.

\newcommand{\xvalue}{value}
\newcommand{\xvalues}{values}

\subsection{Labels}
A \xvalue{} is associated with a label.
The label is a set of privacy policies.
When the \xvalue{} flows through the system, all the policies need to be obeyed.
This means that the effective set of \principals{} able to read the \xvalue{} is the intersection of all policies in a label.

A privacy policy is represented as an owner of some \xvalue{} and a set of readers.
The syntax is: $\policy{\text{<\emph{owner}>}}{\text{<\emph{reader}>}}$.
The owner is the \principal{} who owns the \xvalue{} that was used for constructing the \xvalue{}.
\mikael{Forstår ikke dennne sætning.}
\mikkelin{Nej, jeg mangler også en forklaring for evt at kunne skrive den om.}
The readers are the \principals{} allowed by the owner.
The owner is implicitly allowed to read his own data.
If we want a \principal{} $p$ that should not allow any other readers we provide an empty reader list: $\policy{p}{}$.

If we have policy $K$ and label $L$, we have following notation:
\begin{itemize}
\item $K \in L$ -- policy $K$ is a part of label $L$
\item $\textbf{o}(K)$ -- the owner of policy $K$
\item $\textbf{r}(K)$ -- the set of readers of policy $K$
\end{itemize}

Labeled \xvalues{} are only released by the consensus of all the owners and can only be read by the readers.
If one or several privacy policies are added to a label it restricts the access to the \xvalue{}.
A label with no policies  means that all readers are allowed.

If a \principal{} is not among the owners of a label, it is the same as if it was added as a privacy policy with all posible readers, implicitly indicating that this owner has no preference of who reads the \xvalue{}.

\begin{example}{Redundant owner}
In a system with owners $o_1, o_2, o_3$ and readers $r_1, r_2, r_3$, the following labels are defined:
$$L_1 = \{\policy{o_1}{r_1,r_2};\; \policy{o_2}{r_2, r_3}\}$$
$$L_2 = \{\policy{o_1}{r_1,r_2};\; \policy{o_2}{r_2, r_3};\; \policy{o_3}{r_1, r_2, r_3}\}$$
Since the policy of $o_3$ is to allow everyone to read it introduces no restriction to the label.
Therefore we say that the effective reader set of the two labels are equal.
\end{example}

A label can contain several privacy policies for the same \principal{}.
These are enforced just as other policies.

When examining labels, it is interesting to know if one label is at least as restrictive as another label.
A label has this property if it enforces all the policies/restrictions of the other label.
Formally this can be defined as:

\begin{definition}\label{dlm:def:restrict}
  Let $L_1$ and $L_2$ be labels.
  Then if
  $$\forall j \in L_1 : \exists k \in L_2 : o(k) = o(j) \wedge r(k) \subseteq r(j)$$
  $L_2$ is at least as restrictive as $L_1$, and we write $L_1 \sqsubseteq L_2$.
  \mikkelin{Copied this definition from below. Shouldn't it be $r(j) \subseteq r(k)$?}
\end{definition}

This relation is applied throughout the following.

\subsection{Rules}
This section specifies a set of rules that apply when manipulating labels to avoid information leakage.
The modification of labels is what makes it possible to define how data flows through a system.

\paragraph{Label join}
When deriving a \xvalue{} from two \xvalues{}, the label $L_{12}$ of the derived \xvalue{} must reflect the labels $L_1$ and $L_2$ of its sources, being at least as restrictive as both of them.
This is achieved by taking the union of the two labels.

\begin{definition}
  The join of two labels $L_1$ and $L_2$, denoted as $L_{12} = L_1 \sqcup L_2$, is defined as
  $$L_1 \sqcup L_2 = L_1 \cup L_2$$
  Note that, since $L_1 \subseteq L_{12}$ and $L_2 \subseteq L_{12}$, we have that $L_1 \sqsubseteq L_{12}$ and $L_2 \sqsubseteq L_{12}$.
\end{definition}

The join rule will ensure that any joining of values will also result in a joining of labels, so that any combination of rules will be upheld.

\paragraph{Relabeling by declassification}
The owners of the labels control their data, but sometimes policies are overly restrictive and one wants to relax them.
This can be used to sanitize \xvalues{} whose security policies, in respect to one ore more owners, have changed throughout the run of a program.
This works in opposition to restricting rules, which is used to ensure information is not leaked, whereas this is enabling intended leaking (of sanitized information).

The authority is a set of \principals{} which is the authority of the process of declassification.
If a process has the authority of a \principal{}, the actions of the process are permitted.
This means that if a \principal{} is in the authority set this can be declassified.

\begin{definition}
  Let $L_A = \{\policy{p_0}{}, \policy{p_1}{}, \cdots, \policy{p_n}{}\}$ be the authority label with \principals{} $p_0, p_1, \cdots, p_n$.
  Let $L_1$ and $L_2$ be labels.
  Then if $$L_1 \sqsubseteq (L_2 \sqcup L_A)$$
  $L_1$ may be declassified to $L_2$.
\end{definition}
\stefan{Jeg synes desclassification er rodet. Hvor kommer authority fra, og hvordan indikerer man det? Definition 2.10 ser ud til at sige det samme flere gange (noget med  $L_1 \sqsubseteq L_2 \sqcup L_A$, men jeg er ikke sikker på om det er forskellige pointer)}
\mikkelin{Jeg har skrevet det om, så det ikke står dobbelt.
Authority er beskrevet over definitionen - skal vi have det mere med?}

\paragraph{Relabeling by Restriction}
When a \xvalue{} is read from a variable it has the same label as the variable.
When a \xvalue{} is stored in a variable the label of the \xvalue{} is forgotten.

To illustrate the process of relabeling, the following example describes a simple assignment from one variable to another.

\begin{example}{Assignment}
  This example describes two variables $a = 4$ and $b = ...$.
  We will disregard the current value of $b$ and focus on the relabeling that takes place when processing $$b = a$$

  First we read the \xvalue{} of $a$ and assign to it the label of $a$.
  Afterwards the \xvalue{} is copied to $b$.
  This copy is assigned the label of $b$.
\end{example}

To avoid leakage the label of the variable must be at least as restrictive as the \xvalue{} being assigned to it.
This means that in the example above $b$ is required to be at least as restrictive as $a$.

\subsection{Acts-for relation}
A \principal{} in DLM can represent a user in the system, a group of users, or roles.
As labels on slots are immutable, in order to have more dynamic security policies we have that \principals{} are able to \textit{act for} other \principals{}.
This means that a given \principal{} in effect has every right that the \principal{} that it acts for has.

This is also called a \principal{} hierarchy.
Formally, we use the $\succeq$-operator when representing acts for\footnote{$\succeq$ is reflexive and transitive and not anti-symmetric.} such that $x \succeq y$ declares that $x$ acts for $y$.
A \principal{} hierarchy $P$ is a set of ordered pairs of \principals{}.
So if we have $P \vdash x \succeq y$ it means that $(x,y) \in P$.
\mikkelin{Jeg ved ikke hvad det sidste her betyder.
Er det bare en mængde med alle acts-for relationerne?}

Below are a couple of examples of acts-for relations:
\begin{example}{Acts-for relation}
  Bob can have Amy act on his behalf by providing the acts for relation: $Amy \succeq Bob$.
  This implies that Amy can \textit{act for} Bob and that any security policies that apply to Bob will in effect also apply to Amy.
\end{example}

\begin{example}{Acts-for relation in groups}
  Given a \principal{} $Admin$ that have certain system access we can represent roles/group members using the acts-for relation.
  The relation $Amy \succeq Admin$, signifies that Amy is an administrator and therefore can act on behalf of the administrator group. 
\end{example}

\subsection{Channels}
The model contains two types of channels; input and an output channels.
These channels provide the means of communication with outside systems.

As information can be leaked through these channels, therefore they have a label associated to them.
When a \xvalue{} enters the system through an input channel, the \xvalue{} gets the label of the input channel.
If a \xvalue{} is written to an output channel it can only be done if the label of the output channel is at least as restrictive as the label of the \xvalue{}.

\subsection{Smart Meter System Example}\label{dlm-example}
To demonstrate how the DLM could be used in a smart meter context, we provide an example below.
The example describes a smart meter system while identifying the various \principals{} (actors) and labeled objects.
This is similar to the provided motivating examples presented in \citet{myers1997decentralized}.

\begin{figure}[h]
\centering
\includegraphics[width=\textwidth]{figures/dlm_sm_example.pdf}
\caption{Abstract smart meter system with DLM principals and labelled objects}
\label{dlm:sm_example}
\end{figure}

In \cref{dlm:sm_example} can be seen the abstract smart meter system, with principals (ovals), labelled objects (rectangles), and trusted agent (double oval).
A trusted agent is a principal that is trusted by the owner of the data to modify the data and associated policies.
Arrows display how information flows.
There are two overall components to this system: the Data Hub and the Smart Meter.
The rectangles used for these components signify only a grouping of objects.

Initially, consumers own the data contained in their smart meter, most importantly the raw consumption data.
$PE$ extracts the consumption data and transforms it into a granularity that does not reveal information about the consumer.
In order for the distributor and electrical company to carry out their tasks, balancing the electrical grid and correctly billing, they have access to the transformed power consumption data.
Inside the Data Hub, the distributor receives ownership of all consumption data\footnote{This is actually unclear, but not important to discuss at this point}, but for the power consumption data for each individual consumer $C_i$ the only allowed additional reader is the current electrical company of $C_i$: $EC_j$.
In case $C_i$ changes to a new electrical company $EC_k$, the read access must be removed from $EC_i$ and given to $EC_k$.

The different electrical companies will also provide their current tariffs to the data hub, so that they can be sent to the smart meters and for anyone to see the pricing of available electrical companies in case they want to switch supplier.
The label is empty so that anyone can read the tariffs.
