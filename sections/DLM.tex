%!TEX root=../master.tex

\newcommand{\policy}[2]{\ensuremath{#1\!:\;#2}}

\section{Decentralized Label Model}
The Decentralized Label Model (DLM) is an information flow control model.
As the name suggests it revolves around labels (similar to previous security models), however, DLM is decentralized.
This means that it can be applied to systems with no trusted third party, or even any trust throughout the system.
By attaching security labels to objects in code it can be controlled how information should be shared throughout the code and at code-endpoints (input/output to/from other programs).
It is possible to do both static and run-time checking of labels.
The final major point of DLM is that it is formalized and even though no implementation is supplied, it should be applicable to other existing programming languages.

\newcommand{\xvalue}{value}
\newcommand{\xvalues}{values}
This section describes DLM both formally and informally with small examples.
\mikkelin{Jeg har svært ved at finde den uformelle beskrivelse i teksten.}

\subsection{Definition of terms}
\mikkelin{\#44 Acts-for skal flyttes til der hvor det bruges første gang (Limitations som det er nu)}
\mikkelin{\#44 Privacy policy skal flyttes til Labels}
This section defines and explains some of the terms used to explain the DLM.

\paragraph{Acts for relation}
The acts for relation, $\rightarrow$, can be used to model groups and roles.
To model two \principals{}, Amy and Bob, that act for a group of programmers: $Amy \rightarrow programmers$ and $Bob \rightarrow programmers$.
Or if Amy has different roles: $Amy \rightarrow Amy\_programmer$ and $Amy \rightarrow Amy\_admin$.
\mikkel{\#45 Vi mangler en beskrivelse af hvad formålet med acts for er (også bedre forklaring af eksemplerne) og hvad det betyder i DLM.}

\stefan{\#46 brug $\succeq$ i stedet for $\rightarrow$ }

This is also called a \principal{} hierarchy.
Formally we use the $\succeq$-operator when representing acts for.\footnote{$\succeq$ is reflexive and transitive and not anti-symmetric}
For instance $x \succeq y$ means $x$ acts for $y$.
A \principal{} hierarchy $P$ is a set of ordered pairs of \principvals{}.
So if we have $P \vdash x \succeq y$ it means that $(x,y) \in P$.

\paragraph{Privacy policy}
A privacy policy is represented as an owner of some \xvalue{} and a set of readers.
The syntax is: $\policy{\text{<\emph{owner}>}}{\text{<\emph{reader}>}}$.
The owner is the \principal{} who owns the \xvalue{} that was used for constructing the \xvalue{}.
The readers are the \principals{} allowed by the owner.
The owner is implicitly allowed to read his own data.
If we want a \principal{} $p$ that should not allow any other readers we use the following syntax: $\policy{p}{}$.

\subsection{Labels}\bruno{Måske lidt roddet afsnit?}
A \xvalue{} is associated with a label.
The label is a set of privacy policies.
When the \xvalue{} flows through the system, all the policies need to be obeyed.
This means that the effective set of \principals{} able to read the \xvalue{} is the intersection of all policies in a label.


Labeled \xvalues{} are only released by the consensus of all the owners and can only be read by the readers.
If one or several privacy policies are added to a label it restricts the access to the \xvalue{}.
A label with no policies  means that all readers are allowed.

If a \principal{} is not among the owners of a label, it is the same as if it was added as a privacy policy with all posible readers, implicitly indicating that this owner has no preference of who reads the \xvalue{}.

\begin{example}{Redundant owner}
In a system with owners $o_1, o_2, o_3$ and readers $r_1, r_2, r_3$, the following labels are defined:
$$L_1 = \{\policy{o_1}{r_1,r_2};\; \policy{o_2}{r_2, r_3}\}$$
$$L_2 = \{\policy{o_1}{r_1,r_2};\; \policy{o_2}{r_2, r_3};\; \policy{o_3}{r_1, r_2, r_3}\}$$
Since the policy of $o_3$ is to allow everyone to read it introduces no restriction to the label.
Therefore we say that the effective reader set of the two labels are equal.
\end{example}

When a label interprets its privacy policies it understands it as a set of flows.
A flow $(o,r)$ is a flow of information from owner $o$ to reader $r$.
If a label contains this flow it means that the owner allows the reader to read the \xvalue{}.
If we have three \principals{} $A$, $B$, $C$ and a label: $\{\policy{A}{B}; \ \policy{C}{}\}$.
This means that $A$ allows its \xvalue{} to be read by $B$, $B$ allows every reader and $C$ allows no readers.
Therefore the flow set of the label is: $\{(A,B), (B,A), (B,B), (B,C) \}$.
\stefan{\#47 Skal omskrives så det forklarer hvordan MAN skal fortolke en label, i stedet for hvordan en laber fortolker sig selv}
\stefan{\#47 Beskrivelse af hvad man skal bruges flows til}


A label can contain several privacy policies for the same \principal{}.
These are enforced just as other policies.

Some further notation:
We have policy $K$ and label $L$.
\begin{itemize}
\item $K \in L$, policy $K$ is a part of label $L$
\item $\textbf{o}(K)$ or $\textbf{o}K$, the owner of policy $K$
\item $\textbf{r}(K)$ or $\textbf{r}K$, the set of readers of policy $K$
\end{itemize}
\stefan{\#48 Omskriv til o(k) notation i hele afsnittet}
\mikkelin{\#49 Skal flyttes til starten af Labels (der hvor policies introduceres)}

\subsection{Rules}
This section contains the rules that need to be followed when manipulating labels to avoid information leakage.

\paragraph{Label join}
When deriving a \xvalue{} from two \xvalues{}, the derived \xvalue{}s label must reflect its sources, so it has to be at least as restrictive.
Thus the label assigned to the derived \xvalue{} is the union of the two source labels.
This gives us the following join rule:
\begin{definition}
  $L_1 \sqcup L_2 = L_1 \cup L_2$
  \mikkelin{\#50 Beskrivelse af $\sqcup$ / join}
\end{definition}


\paragraph{Relabeling by declassification}
\stefan{\issue{51} Intro og forklaring af hvad det skal bruges til}
The owners of the labels control their data, but sometimes policies are overly restrictive and one wants to relax them.

The authority is a set of \principals{} which is the authority of the process of declassification.
If a process has the authority of a \principal{}, the actions of the process are permitted.
This means that if a \principal{} is in the authority set this can be declassified.

A label $L_1$ can be relabeled to $L_2$ if $L_1 \sqsubseteq L_2 \sqcup L_A$ where $L_A$ is the authority.
The authority is a label $L_A$ of the form $\{p: \ \}$ for every \principal{} in the current authority.

$L_1$ can be relabelled to $L_2$ if $L_1 \sqsubseteq L_2 \sqcup L_A$ is true.
\begin{definition}
    $$\frac{L_1 \sqsubseteq L_2 \sqcup L_A}{L_1 \text{ may be declassified to } L_2}$$
\end{definition}

\paragraph{Relabeling by Restriction}
This rule is for when one is doing an assignment of a \xvalue{} into a variable.
When a \xvalue{} is read its label is relabelled to the variable and forgotten.
\mikkel{Hvad betyder det at en label er relabelled til en variabel?}
To avoid leakage the label of the variable must be at least as restrictive.
For instance if we have two labels, $L_1$ and $L_2$, $L_1 \sqsubseteq L_2$ means that $L_2$ is more restricted than $L_1$.

So a restriction is when the new label guarantees to enforce all of the policies from the old label.
If we have policy $J$ in $L_1$ it is guranteed by another policy $K$ if they have the same owner and $rK$ is a subset of $rJ$.
\begin{definition}
$$\frac{\forall (J \ \in \ L_1) \exists (K \ \in \ L_2)(oK = oJ \ \wedge \ rK \subseteq rJ)}{L_1 \sqsubseteq L_2}$$
\end{definition}
\bruno{definition of the relabeling rule - skal den være her?}

\subsection{Limitations}
The rule of restriction is too strict as it is only possible to remove readers and add policies \stefan{hvorfor er dette tilfældet?} and therefore prevents valid relabelings.
There are three possible valid relabelings which are not possible at the moment:
\mikkel{\#44 Acts for burde blive forklaret heromkring sådan at det passer med det emne det relaterer sig til.}
\begin{itemize}
\item Adding readers - if we have $r' \succeq r$ and $r$ is a reader of some owner, $'r$ should be too.
\item Replacing owners - if we have $o' \succeq o$ then it should be possible to replace $o$ with $o'$.
\item Self-authorization - every \principal{} that acts for an owner of a policy should be possible to add as a reader.
\end{itemize}

\subsubsection{Extending the relabeling by restriciton rule}
We need to extend the rule in order to be able to do the things discussed above.

To take the \principal{} hierarchy into account when interpreting a label an interpretation function $\textbf{X}$ is introduced.
For now\mikkel{Ændres det senere?} $\textbf{X}$ takes a label as argument and the set of \principals{} $P$ as an implicit argument.\mikkel{Hvad er et implicit argument?}
Informally this function makes sure that if a reader or owner that acts for some \principal{} $p$ - that $p$ is included in the label interpretation.
\mikkel{Jeg forstår ikke den her sidste sætning. Det er som om der mangler noget.}

The set of flows for a label need to satisfy two constraints for the given principal hierarchy.
One for readers and one for owners.
The following is the shows the reader constraint:
If we have $(o,r)$ and $r' \succeq r$ then $(o,r)'$ must also be in the set of flows.
Formally ($\rightarrow$ used for implication):
\begin{center}
  $r' \succeq r \ \wedge \ (o,r) \in \textbf{X}L \rightarrow (o,r') \in \textbf{X}L$
\end{center}
And the owner constraint:
If we have $(o,r)$ and $o' \succeq o$ and relabel to $(o',r)$, so a owner gets replaced by an owner that acts for it.
\begin{center}
  $o' \succeq o \ \wedge \ (o',r) \notin \textbf{X}L \rightarrow (o,r) \notin \textbf{X}L$
\end{center}

The function \textbf{R} is essentially adds all readers that can be found by looking at the acts for relations.
The \textbf{O} function converts a label into a set of flows and removes the ones that are not allowed.
Definition of the interpretation function $\textbf{X}$:
\begin{definition}
  $\textbf{X}L \ = \ \textbf{OR}L \ = \ \{(o,r)|\forall (I \in L) \textbf{o}I \succeq o \rightarrow [r \succeq \textbf{o}I \vee \exists (r' \in \textbf{r}I) r \succeq r'] \}$
\end{definition}

By using \textbf{X} we can verify a relabeling by applying the following correctness check.
\begin{definition}
  $\frac{\textbf{X}(L_1, P)\subseteq \textbf{X}(L_2,P)}{\text{Relabeling from } L_1 \text{ to } L_2 \text{ is safe in } P }$
\end{definition}
\mikkelin{Jeg synes alle definitionerne i ovenstående er for lidt forklarende.
Det er for lidt information man får når der kun er en ligning/et udtryk.}

\subsection{Channels}
The model contains two channels, an input and an output channel.
Information can be leaked through these channels, therefore they have a label associated to them.
When a \xvalue{} enters the system through an input channel, the \xvalue{} gets the label of the input channel.
If a \xvalue{} is written to an output channel it can only be done if the label of the output channel is at least as restrictive as the label of the \xvalue{}.
\bruno{Example?}
\mikkelin{Jeg synes vi skal lave et eksempel (fx noget pseudo kode) og så bruge det fra start til slut.}

\subsection{Static checking}
\bruno{Udvider correctnes check'et så det kan bruges til statisk check}
